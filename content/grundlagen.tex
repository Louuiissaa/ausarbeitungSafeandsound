\chapter{Grundlagen}

\section{Raspberry Pi}

\section{Node Red}

\section{SQLite Datenbank}

\section{Regelbasiertes System} %Louisa
Im heutigen Alltag werden kontinuierlich Daten erfasst, um auf dessen Basis Informationen zu erhalten. Daten werden aus unterschiedlichen Gründen gesammelt, meist steht jedoch hinter der Datensammlung der Wunsch einen Mehrwert aus den Daten zu schaffen. Um Daten nutzen zu können, müssen diese kanalisiert, strukturiert und archiviert werden \cite{managermagazin:datensammlung}. Regelbasierte Systeme (kurz RBS), die eine Sonderform eines Expertensystems bilden \cite{ieee:ruleBasedSystemAndNetworks}, werden für die Datenstrukturierung eingesetzt. Ursprünglich stellten diese Systeme lediglich codierte Problemlösung mit dem Wissen von menschlichen Experten dar \cite{Hayes-Roth:1985:RS:4284.4286}. Im Laufe der Zeit hat sich diese Definition erweitert. Neben Regeln, die von einem Experten gesetzt wurden, stellen regelbasierte Systeme auch Regeln auf, die auf Basis alter Datensätze erstellt wurden. Es wird aus diesem Grund bei der Erstellung eines solchen Systems unterschieden zwischen einem Experten basierten und einem Daten basierten Design. Gerade im Zuge der stark ansteigenden Datenmengen findet man regelbasierte Systeme mit einem Daten basiertem Design immer öfter im Einsatz \cite{ieee:ruleBasedSystemAndNetworks}.\\
Die problemlösenden Methoden werden in einem RBS durch Situation-Aktion Regeln abgebildet. Um sinnvolle Lösungen zu Problemstellungen zu erlangen, sollten die folgenden fünf Punkte beim Aufstellen von Regeln beachtet werden.\\
Das Wissen sollte in Wenn-Dann-Regeln aufgestellt werden. Die Kondition wird im ersten Teil der Regel definiert, dem Wenn-Teil. Dabei können mehrere Konditionen aneinander geknüpft werden. Im zweiten Teil der Regel, dem Dann-Teil, wird die Aktion beschrieben, die nach der Erfüllung der Kondition ausgeführt wird. Es kann davon ausgegangen werden, dass wenn die Kondition erfüllt ist, auch die Aktion erfüllt ist. Eine RB kann aus diesem Grund als Interpreter von Wenn-Dann-Regeln gesehen werden\cite{oracle:JREAPI}.\\
Neben der Darstellung einer Regel durch die Wenn-Dann Form, können Regeln auch durch Entscheidungstabellen abgebildet werden \cite{HSAugsburg:RuleEngine}. Entscheidungstabellen verknüpfen Aktionen mit Bedingungen \cite{itwissen:entscheidungstabelle}. Der Unterschied zu den zuvor beschriebenen Wenn-Dann-Regeln, ist die formale Aufstellung der Regeln in Form von einer Tabellenstruktur.\\
Als zweiten Punkt, den man beim Aufstellen von Regeln beachten soll, ist die Steigerung der Kompetenz des RBS proportional zur Wissenserweiterung. Wenn diese Steigerungen nicht proportional zueinander verlaufen, wird das RBS auf lange Sicht nicht den gewünschten Mehrwert mit sich bringen.\\
Des Weiteren sollte das RBS durch Kombination von Regeln auch komplexere Problemstellungen lösen. Dabei werden die Ergebnisse der Regelausführung sinnvoll kombiniert. Um Probleme bestmöglich lösen zu können, sollten RBS die optimale Reihe von Regeln bestimmen und anschließend ausführen. Die nach Durchführung der Regel erhaltene Lösung zur Problemstellung sollte vom System anschließend in natürliche Sprache oder in Aktionen, die für den Menschen verständlich sind, übersetzt werden.\\
Genutztes Wissen innerhalb eines RBS wird als problemlösendes Wissen bezeichnet und besteht aus verschiedenen Formen von Informationen. So können Informationen durch Beobachtungen gesammelt werden, aber auch durch Abstrahierung, Generalisierung oder Kategorisierung von gegebenen Daten. Neben reinen Daten werden auch Verknüpfungen von Daten als relevante Informationen angesehen, sowie Strategien zur Datengewinnung \cite{Hayes-Roth:1985:RS:4284.4286}.\\
Bei der Erstellung eines RBS muss jedoch beachtet werden, dass ein geeignetes System stetig angepasst und verbessert werden muss und vor allem durch das hinzukommende Wissen optimalere Lösungen ermitteln kann \cite{Hayes-Roth:1985:RS:4284.4286}.




