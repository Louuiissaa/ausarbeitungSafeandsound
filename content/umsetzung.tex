\chapter{Umsetzung} \label{sec:Umsetzung}

\section{Architektur}
Auf Basis der Grundlagen und Analyse wird die Architektur des Systems erstellt.
In der Architektur werden alle wichtigen Entscheidungen, welche für die
Entwicklung des Systems relevant und von technischer Natur sind, berücksichtigt. Die Architektur dient als Grundlage für die darauffolgende Umsetzung.

\section{Raspberry Pi}
\subsection{Sensor Adapter}

\subsection{Persistenter Speicher}

\section{Grafische Benutzeroberfläche} %Louisa
Als grafische Benutzerschnittstelle für das in dieser Arbeit erläuterte System wird eine ortsunabhängige Lösung entwickelt (F-10.2). Um eine Ortsunabhängigkeit gewährleisten zu können, bietet es sich an ein mobiles Endgerät zu entwickeln. Es wurde  entschieden als Benutzerschnittstelle eine Android App zu nutzen, aus dem Grund, dass Android den höchsten Endkundenabsatz besitzt \cite{statista:marktanteileandroid} und dadurch eine größere Maße erreicht werden kann.
\subsection{Rule Engine}
Die Rule Engine wird nach den Anforderungen aus Kapitel \ref{sec:Anforderungen} entwickelt. Demnach muss die Rule Engine dem Nutzer ermöglichen dynamisch Regeln hinzufügen zu können oder wieder zu löschen (F-30.1). Grenzwerte oder Wertebereiche kann der Nutzer für die erfassten Datentypen (F-10) selbst bestimmen (F-30.2). Des Weiteren soll die Rule Engine zeitliche Aspekte in den Regeldefinitionen einbinden können. Der Nutzer kann demnach zeitliche Intervalle festlegen, in denen ein Wert oder Wertebereich gehalten werden soll. Die Definition von Datenabhängikeiten kann von der Rule Engine verarbeitet werden, um optimalere Aktionen auslösen zu können (F-30.5).\\
Es werden mögliche Rule Engines recherchiert, die die gestellten Anforderungen erfüllen. Dabei wurden nur Rule Engines betrachtet, die in Android genutzt werden können, sowie welche mit regelmäßigen Releases. Aus der Recherche haben sich folgende Rule Engine Bibliotheken ergeben:
\begin{itemize}
\item Easy Rules \cite{github:easyrules}
\item OpenRules \cite{openrules}
\item RuleBook \cite{github:rulebook}
\end{itemize}
Diese Rule Engines bieten gute Möglichkeiten Regeln zu erstellen und diese zu überprüfen. Der entscheidene gemeinsame Nachteil dieser Rule Engines ist die Implementierung der Regeln. Durch die Recherche konnte keine Engine gefunden werden, die eine dynamische Erstellung von neuen Regeln über die Oberfläche ermöglichen. Die Definition neuer Regeln durch den Nutzer ohne den Quellcode zu verändern ist jedoch eine essentielle Anforderung. Aus diesem Grund wurde sich dafür entschieden, eine Rule Engine selbst zu implementieren.\\
Bei der Implementierung der Rule Engine wurden die Ergebnisse der Recherche miteinbezogen. Vor allem an die Struktur der Regeln zu ihren Wenn- und Dann-Teilen der Regel Engine RuleBook wurde bei der Implementierung betrachtet.


- Warum keine RuleEngine Bibliothek nutzen?
	-mögliche Rule Engines: es wurden nur RuleEngines betrachtet, die in Android genutzt werden können
		- Easy Rules
		- OpenRules
		- Dredd
		- RuleBook
			-Vorteile:  
		- Nachteile bzw. Vorteile von diesen Engines
- Warum keine bekannt Rule Sprache nutzen um Regeln zu definieren? - RuleML

\subsection{Daten Anzeige}