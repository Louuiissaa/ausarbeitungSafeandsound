\chapter{Ausblick}

Die Anwendung kann zu einer Home Automation Steuerung entwickelt werden. Dafür könnten automatisierte Vorgänge im Raum gesteuert werden. Ein beispielhafter Use Case ist, Fenster automatisch schließen und öffnen zu können. Eine ungewünschte Gaskonzentration, wird von der Rule Engine erkannt und diese löst über den Raspberry Pi eine Aktion bei dem Akteur aus. In diesem Fall wäre der Akteur ein Motor am Fenster, der dieses öffnen und schließen kann.\\
Ein weiteres Beispiel könnte die Vorhersage von Daten sein. Dazu müsste basierend aus den existierenden Daten ein Trend analysiert und ermittelt werden, sodass der Anwender mitgeteilt bekommt, wie der Temperaturverlauf in den nächsten Stunden wahrscheinlich sein wird. Um Vorhersagewerte treffen zu können, kann ein Datenmodell konzipiert werden und mit Hilfe eines Skriptes in der Sprache R ausgeführt werden. Die Vorhersagewerte werden in einer extra Spalte der Datentabelle eingefügt.\\
Die Vorhersagewerte bringen vor allem dann einen Mehrwert, wenn die Sensoren zeitweise ausfallen. Über die Vorhersage können dann weiterhin Handlungsvorschläge getroffen werden. Der Vorteil dabei wäre, dass sich die Pfleger weiterhin auf das System verlassen können.\\

Um den Mehrwert der Applikation zu steigern, sollte der Regel Engine eine Timer-Funktion hinzugefügt werden. Durch die Timer-Funktion können Regel hinzugefügt werden auf Basis von Zeitintervallen. Ein Anwendungsfall für den Nutzen eines Timers ist die Statusänderung des Fensters. Ein Nutzer möchte nicht zwingend benachrichtigt werden, wenn ein Fenster geöffnet ist, sondern viel mehr wenn ein Fenster länger als einen gegebenen Zeitintervall geöffnet ist. Durch die Implementierung der Rule Engine mit einem JavaScript Interpreter ist das Hinzufügen eines Timers möglich. JavaScript bietet Bibliotheken für Timer Implementierungen an.\\
