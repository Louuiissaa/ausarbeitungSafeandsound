
\chapter{Ausblick}
Vorhersagen von WErten,
Analyse Verfahren entwickeln -> Gründe finden warum Raum nicht perfekt ist
Software soll über Motoren vorgänge automatisieren -> zu kalt automatisch fenster schließen
Lüften bis gute Werte erzielt werden

Die Anwendung könnte sich in Anbetracht in die Richtung einer Home Automation Steuerung entwickeln. Dafür müsste automatisierte Vorgänge im Raum gesteuert werden können. Beispielweise könnten als Use Case definiert werden, dass über die App basierend auf vom Benutzer definierten Regeln, Nachrichten gesendet werden, die beeinflussen, dass automatisch die Fenster geöffnet werden wenn es im Raum zu schlechte Gaskonzentrationswerte gibt.  
\\Ein weiteres Beispiel könnte die Vorhersage von Daten sein. Dazu müsste basierend aus den existierenden Daten ein Trend analysiert und ermittelt werden, sodass der Anwender mitgeteilt bekommt, wie der Temperaturverlauf in den nächsten Stunden wahrscheinlich sein wird. Um Vorhersagewerte treffen zu können, kann ein Datenmodell konzipiert werden und mit Hilfe eines Skriptes in der Sprache R ausgeführt werden. Die Vorhersagewerte werden in einer extra Spalte der Datentabelle eingefügt.\\
Die Vorhersagewerte bringen vor allem dann einen Mehrwert, wenn die Sensoren zeitweise ausfallen. Über die Vorhersage können dann weiterhin Handlungsvorschläge getroffen werden. Die Pfleger können sich weiterhin auf das System verlassen.\\
Um den Mehrwert der Applikation zu steigern, sollte der Regel Engine eine Timer-Funktion hinzugefügt werden. Durch die Timer-Funktion können Regel hinzugefügt werden auf Basis von Zeitintervallen. Ein Anwendungsfall für den Nutzen eines Timers ist die Statusänderung des Fensters. Ein Nutzer möchte nicht zwingend benachrichtigt werden, wenn ein fenster geöffnet ist, sondern viel mehr wenn ein Fenster länger als einen gegebenen Zeitintervall geöffnet ist. Durch die Implementierung der Rule Engine mit einem JavaScript Interpreter ist das Hinzufügen eines Timers möglich. JavaScript bietet Bibliotheken für Timer Implementierungen an.\\