\chapter{Ausblick}
Durch die verschiedenen Komponenten ist es möglich schnell weitere Sensoren in das System hinzufügen.

Die Anwendung könnte sich in die Richtung einer Home Automation Steuerung entwickeln. Dafür müsste automatisierte Vorgänge im Raum gesteuert werden können. Beispielweise könnten als Use Case definiert werden, dass über die App, basierend auf vom Benutzer definierten Regeln, Nachrichten an den Raspberry Pi gesendet werden, die beeinflussen, dass Motoren automatisch die Fenster öffnen, wenn es im Raum zu schlechte Gaskonzentrationswerte gibt.  
\\Ein weiteres Beispiel könnte die Vorhersage von Daten sein. Dazu müsste basierend aus den existierenden Daten ein Trend analysiert und ermittelt werden, sodass der Anwender mitgeteilt bekommt, wie der Temperaturverlauf in den nächsten Stunden wahrscheinlich sein wird.
\chapter{Ausblick}
Um den Mehrwert der Applikation zu steigern, sollte der Regel Engine eine Timer-Funktion hinzugefügt werden. Durch die Timer-Funktion können Regel hinzugefügt werden auf Basis von Zeitintervallen. Ein Anwendungsfall für den Nutzen eines Timers ist die Statusänderung des Fensters. Ein Nutzer möchte nicht zwingend benachrichtigt werden, wenn ein fenster geöffnet ist, sondern viel mehr wenn ein Fenster länger als einen gegebenen Zeitintervall geöffnet ist. Durch die Implementierung der Rule Engine mit einem JavaScript Interpreter ist das Hinzufügen eines Timers möglich. JavaScript bietet Bibliotheken für Timer Implementierungen an.\\

