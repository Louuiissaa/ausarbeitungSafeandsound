\chapter{Einleitung}

\section{Motivation}
Menschen möchten ihre Umgebung verstehen und überwachen, ob aus reiner Neugier, Kontrolle oder dem Bedürfnis nach mehr Sicherheit.
In Folge der Digitalisierung werden mehr und mehr Lösungen zur Verfügung gestellt, um diese Bedürfnisse der Menschen zu stillen. Bewohner können sich selbst, sowie ihre Sachgegenstände und ihren Lebensraum mit der Hilfe dieser Lösungen schützen und kontrollieren. Mit wenig Aufwand ist es bereits möglich die Haustür per Bildübertragung zu überwachen oder Bewegungen im Heim zu erkennen.

Je nach Besitz, sozialem Umfeld und persönlichem Interesse unterscheiden sich jedoch die Kontroll- bzw Überwachungsbedürfnisse der Menschen. Zu erhaltene Lösungen sind meist starr von den verknüpfbaren Sensoren und können nicht übergreifend die jeweiligen Anforderungen der Bewohner erfüllen.
Eine individuelle Zusammenstellung von Überwachungskomponenten und eine übergreifende Auswertung der resultierenden Werte sind jedoch für viele Bewohner von großem Nutzen. Beispielsweise kann ein Nutzer den optimalen Zeitpunkt und Zeitraum zum Lüften ermitteln aus den Werten von Temperatur, Luftgas und Fenster Sensoren. Es kann erkannt werden, dass wenn die CO\textsubscript{2} Konzentration ansteigt, das Fenster jedoch geschlossen ist, das Öffnen des Fensters eine positive Auswirkung auf die Luftqualität erzeugt. Eine kalte Außentemperatur kann jedoch bei zu langem Öffnen des Fensters die Raumtemperatur negativ beeinflussen. Deshalb wird nach erreichen einer akzeptaplen CO\textsubscript{2} Konzentration die Empfehlung gegeben, das Fenster wieder zu schließen.

Ein Zusammenspiel unterschiedlicher Sensoren kann in vielen Anwendungsbereichen eine positive Auswirkung auf die Lebensqualität oder auf die Sicherheit der Bewohner bewirken.
\section{Ziel}
Ziel dieser Arbeit ist es eine mögliche Lösung zu geben Sensordaten zu erhalten und aus den ermittelten Sensordaten Verknüpfungen herzustellen. Sensordaten sollen übergreifend in Relation gesetzt werden. Durch mögliche Abhängigkeiten können Nutzern Handlungsempfehlungen zur Verfügung gestellt werden.

Des Weiteren soll ein Ausblick auf eine mögliche Erweiterung einer vorausschauenden Datenverarbeitung gegeben werden. So kann beispielsweise der Fall der Luftqualität schon im Vorhinein erkannt werden, sodass Gegenmaßnahmen frühzeitig getroffen werden.
\section{Aufgabenstellung}
\section{Struktur der Arbeit}
Um das Ziel dieser Arbeit zu erreichen, wird die Arbeit in drei Phasen aufgeteilt. In der ersten Phase wird ermittelt mit welchen Technologien und Lösungen die verschiedenen Sensoren angesprochen und ausgelesen werden können, um anschließend die relevanten Werte in einer Benutzerschnittstelle zu präsentieren.

In der nächsten Phase sollen Zusammenhänge zwischen den Ergebnissen der Sensoren erkannt werden, um Regeln zu definieren, die dem Bewohner ermöglichen ein Optimum zu erreichen. Je nach Sensor und seiner Funktionalität weicht ein gewünschtes Ziel der Überwachung voneinander ab. Eben diese Ziele und gewünschten Optimums der Sensoren müssen definiert und zwischen den Sensoren verknüpft werden.

Aufbauend auf die vorausgehenden Phasen wird in der dritten Phase untersucht, inwiefern Sensordaten vorhergesagt werden können. Durch das Vorhersagen von Daten können präziser und effizienter Maßnahmen vom Nutzer getroffen werden.