\chapter{Einleitung}

\section{Motivation}
Menschen möchten ihre Umgebung verstehen und überwachen, ob aus reiner Neugier, Kontrolle oder dem Bedürfnis nach mehr Sicherheit. Vor allem wenn es um einen wichtigen Menschen, ein Lebewesen oder eine Sache mit persönlichem Wert geht, möchte man dessen Umgebung optimal setzen. Um eben diesen ``perfekten'' Raum zu gewährleisten, muss die Umgebung kontrolliert, analysiert und gegebenenfalls daraus Vorsichtsmaßnahmen ergriffen werden.

In dem ``perfekten'' Raum soll die Temperatur, die Luftfeuchtigkeit, das Luftgasgemisch, sowie die Lichtverhältnisse optimal gesetzt werden. Der Nutzer möchte oftmals keinen festen Richtwert setzen, sondern einen Werteverlauf. Zum Beispiel sollte bei der Umgebungsanalyse einer Pflanze kein konstanter Lichteinfluss im Raum sein, sondern ein Wertverlauf nach Zeit.\\
Eine mobile Schnittstelle lässt den potentielle Besitzer eines ``perfekten'' Raumes ortsunabhängig das zu beobachtende Objekt an sich und seine Umgebung immer im Blick zu haben und bei Unregelmäßigkeiten benachrichtigt zu werden.


\section{Ziel}
Das Ziel dieser Arbeit ist es, durch eine methodische Vorgehensweise einen ``perfekten'' Raum zu schaffen. Der Raum wird dafür als ein regelbasiertes System betrachtet, das die optimale Umgebung für den Inhalt des Raumes bilden wird.\\Dem Nutzer soll eine mobile Schnittstelle zur Verfügung gestellt werden, die ihm dabei hilft, eine visuelle, aber auch eine sensorbasierte Kontrollinstanz für den Raum aufzubauen. Neben der Möglichkeit den Raum zu überprüfen, soll dem Nutzer gleichzeitig aus den erhalteten Sensordaten konkrete Verbesserungsvorschläge gemacht werden.\\
Somit können die Nutzer, auf dieser Benutzerschnittstelle aufbauend, eine effizientere Raumkontrolle und Verbesserung der Umgebung aufsetzen.
Außerdem sollen vom Benutzer Grenzwerte oder ein Werteverlauf selbstständig festgelegt werden können. Der Nutzer wird benachrichtigt, falls der entsprechende Wert sich nicht mehr im Tolleranzbereich befindet. 
Folgende Aufgaben soll die App dabei abdecken:
\begin{itemize}
	\item Anzeige eines Live-Bild einer Kamera
	\item Anzeige der aktuellen Temperatur
	\item Anzeige der aktuellen Luftfeuchtigkeit
	\item Anzeige eines Status, ob ein Fester/Tür geöffnet oder geschlossen ist
	\item Anzeige eines Verlaufs der oben genannten Sensordaten
\end{itemize} 
Die verschiedenen Sensoren werden an einem Raspberry Pi angeschlossen und deren Werte zentral erfasst. Diese ermittelten Werte der Sensoren sollen anschließend über eine Schnittstelle von der App aufgerufen werden können. 
\\
Des Weiteren soll ein Ausblick auf eine mögliche Erweiterung gegeben werden, wie sich Vorhersagen, auf Basis der bestehenden Daten, zu einzelnen Sensor bestimmen lassen und der Benutzer dadurch profitieren kann. So kann beispielsweise im Fall der Luftqualität schon im Vorhinein auf Basis alter Daten eine Verschlechterung erkannt werden, sodass Gegenmaßnahmen frühzeitig getroffen werden können.
\section{Struktur der Arbeit}
- Aufsetzen des Pi's
- Ansprechen und Auslesen der Sensoren
- Speicherung der Daten

Zuerst ist eine Einarbeitung in die Thematik der regelbasierten Systeme, der Sensortechnik und der Nutzung eines Raspberry Pis notwendig. Darauf aufbauend werden die Anforderungen an das zu erstellende System erhoben. Um sinnvolle Anforderungen aufstellen zu können, muss definiert werden, was ein ``perfekter'' Raum ist und dessen Vorteile. Da ein solcher Raum je nach Verwendungszweck des Raumes stark abweichen kann, wird diese Arbeit erstellt am Beispiel ...... Anschließend wird ein übergreifender Anforderungskatalog erstellt. Mit Hilfe des Anforderungskataloges werden anschließend Technologien und Vorgehensweisen zur Erfüllung der Anforderungen verglichen und bewertet.\\
Bei der Auswahl der einzelnen Komponenten um das Ziel dieser Arbeit zu erreichen, wird als Grundlage festgesetzt eine mobile Applikation des Betriebssystemes Android zu erstellen. Auf dadurch entstehende Einschränkungen und Auswirkungen auf die Gesamtarchitektur wird durchgehend gesondert geachtet. 	