\chapter{Einleitung}

\section{Motivation}
Menschen möchten ihre Umgebung verstehen und überwachen, ob aus reiner Neugier, Kontrolle oder dem Bedürfnis nach mehr Sicherheit. Vor allem wenn es um einen wichtigen Menschen, ein Lebewesen oder eine Sache mit persönlichem Wert geht, möchte man dessen Umgebung optimal setzen. Um eben diesen ``perfekten'' Raum zu gewährleisten, muss die Umgebung kontrolliert, analysiert und gegebenenfalls daraus Vorsichtsmaßnahmen ergriffen werden.

Im Rahmen dieser Arbeit wird ein perfekter Raum für Senioren in einem Seniorenheim geschaffen. In diesem ``perfekten'' Raum soll die Temperatur, die Luftfeuchtigkeit, das Luftgasgemisch, sowie die Lichtverhältnisse optimal gesetzt werden. Als Angehörige/r oder als Personal des Seniorenheims möchte man eben diese Werte überwachen, um bei Unregelmäßigkeiten und Abweichungen aus dem Toleranzbereich benachrichtigt zu werden und Maßnahmen ergreifen zu können.\\
Ein/e Angehörige/r möchte in Bezug auf sein Familienmitglied verschiedene Dinge erfahren/ kontrollieren. Eine Motivation dafür ist, dass der Senior sicher in seinem Apartment agieren soll. Durch Mangel an Pflegekräften kann ein regelmäßiges persönliches Kontrollieren von Altenpflegern nicht durchgeführt werden. Zudem würde dadurch dem Seniorenheimbewohner sichtlich in die persönliche Freiheit eingegriffen, wenn in regelmäßigen Abständen Pfleger zur kurzen Kontrolle ins Zimmer kommen würden. Zudem ist die Gesundheit des Senioren von immenser Wichtigkeit. Viele Krankheiten, wie eine Erkältung oder eine Lungenentzündung, lassen sich vermeiden, wenn das Raumklima optimal für den Senior gesetzt ist.\\
Von dieser Motivation aus wird das im nächsten Kapitel beschriebene Ziel angestrebt.
\section{Ziel}
Das Ziel dieser Arbeit ist es, durch eine methodische Vorgehensweise einen ``perfekten'' Raum für Senioren mit Hilfe eines Raspberry Pis zu schaffen. Angehörigen und Pflegern soll ein System zur Verfügung gestellt werden, das ihnen dabei hilft, eine sensorbasierte Kontrollinstanz für den Raum aufzubauen.\\
Um einen optimalen Raum für Senioren zu schaffen, sollen die Raumbedingungen mit elektronischer Hilfe kontrolliert und verbessert werden.
Dafür sollen Angehörige bzw. Pflegekräfte automatisiert benachrichtigt werden, sobald ein manuelles Eingreifen sinnvoll ist. Der Nutzer des Systems soll Regeln definieren können, nach denen dieser dann benachrichtigt wird.
Um sinnvolle Benachrichtigungen geben zu können, müssen diverse Sensoren im Raum platziert werden mit dessen Hilfe die Raumbedingungen überwacht werden können. Es wird ein Algorithmus benötigt, der entscheidet, wann die zuständige Person benachrichtigt wird, mit einem Vorschlag, die Raumbedingungen positiv zu verändern.\\
Zu Beginn muss evaluiert werden, welche Sensoren sinnvolle Daten liefern, um optimale Raumbedingungen für Sensoren gewährleisten zu können. Die Sensordaten müssen erfasst werden und den zentralen Komponenten zur Verfügung gestellt werden.\\
Zentrale Komponenten sind der Raspberry Pi, der Daten verarbeitende Algorithmus, sowie das regelbasierte System.
Für die Erstellung des Algorithmus werden zuerst einfache Soll-Ist Abgleiche der Daten gemacht. Es wird analysiert was geeignete bzw. gewünschte Soll-Werte für das Raumklima sind. Aufbauend auf diesen einfachen Vergleichen werden anschließend Definitionen von komplexeren Zusammenhängen zwischen den Sensordaten definiert. Es wird analysiert, inwiefern mit Messwerten von Sensoren auf andere Sensordaten geschlossen werden können. Aufbauend auf die Bedeutungen der Daten und ihrer Abhängigkeiten untereinander werden Regeln definiert. So kann der Algorithmus sinnvolle Aktionen dem Nutzer vorschlagen, sowie relevante Meldungen ausgeben.\\
Des Weiteren wird eine App auf Android Basis entwickelt. Neben der einfachen Anzeige von Sensordaten, soll der App-Nutzer die gewünschten Grenzwerte oder den Werteverlauf der ermittelten Daten selbstständig festlegen können. Bei der Datenerfassung muss auf Besonderheiten des Datentyps geachtet werden. So besitzt der Wert für die Lichtverhältnisse nicht einen optimalen Wert, sondern viel mehr einen optimalen Werteverlauf.\\
Dabei soll die App die aktuellen Werte anzeigen, grafisch den Verlauf der Daten über die Zeit aufbereiten, sowie die bereits erwähnten Benachrichtigungen kommunizieren.
In der Arbeit soll zudem ein Ausblick auf eine mögliche Erweiterung gegeben werden, wie sich Vorhersagen, auf Basis der bestehenden Daten, zu einzelnen Sensoren bestimmen lassen und der Benutzer dadurch profitieren kann. So kann beispielsweise im Fall der Luftqualität schon im Vorhinein auf Basis alter Daten eine Verschlechterung vorhergesagt werden, sodass Gegenmaßnahmen frühzeitig getroffen werden können.
\section{Struktur der Arbeit}
Zuerst ist eine Einarbeitung in die Thematik der regelbasierten Systeme, der Sensortechnik und der Nutzung eines Raspberry Pis notwendig. Darauf aufbauend werden die Anforderungen an das zu erstellende System erhoben. Um sinnvolle Anforderungen aufstellen zu können, muss definiert werden, was ein ``perfekter'' Raum für Senioren ist und dessen Vorteile. Anschließend wird ein übergreifender Anforderungskatalog erstellt. Mit Hilfe des Anforderungskataloges werden anschließend Technologien und Vorgehensweisen zur Erfüllung der Anforderungen verglichen und bewertet.\\
Bei der Auswahl der einzelnen benötigten Komponenten, wird als Grundlage festgesetzt eine mobile Applikation des Betriebssystemes Android zu erstellen. Auf dadurch entstehende Einschränkungen und Auswirkungen auf die Gesamtarchitektur wird durchgehend gesondert geachtet. 	