\chapter{Einleitung}

\section{Motivation}
Menschen möchten ihre Umgebung verstehen und überwachen, ob aus reiner Neugier, Kontrolle oder dem Bedürfnis nach mehr Sicherheit. Vor allem wenn es um einen wichtigen Menschen, ein Lebewesen oder eine Sache mit persönlichem Wert geht, möchte man dessen Umgebung optimal setzen. Um eben diesen ``perfekten'' Raum zu gewährleisten, muss die Umgebung kontrolliert, analysiert und gegebenenfalls daraus Vorsichtsmaßnahmen ergriffen werden.

Im Rahmen dieser Arbeit wird ein perfekter Raum für Senioren in einem Seniorenheim geschaffen. In diesem ``perfekten'' Raum soll die Temperatur, die Luftfeuchtigkeit, das Luftgasgemisch, sowie die Lichtverhältnisse optimal gesetzt werden. Als Angehörige/r oder als Personal des Seniorenheims möchte man eben diese Werte überwachen, um bei Unregelmäßigkeiten und Abweichungen aus dem Toleranzbereich benachrichtigt zu werden und Maßnahmen ergreifen zu können.\\Eine mobile Schnittstelle lässt zudem zu, den ``perfekten'' Raum für Sensioren ortsunabhängig im Blick zu behalten, sodass die Umgebung optimal für Senioren gesetzt wird.\\Bei der Datenerfassung muss auf Besonderheiten des Datentyps geachtet werden. So besitzt der Wert für die Lichtverhältnisse nicht einen optimalen Wert, sondern viel mehr einen optimalen Werteverlauf.\\
Ein Angehöriger möchte in Bezug auf sein Familienmitglied verschiedene Dinge. Zum einen soll der Senior sicher in seinem Appartment agieren können. Durch Mangel an Pflegekräften kann ein regelmäßiges persönliches Kontrollieren von Altenpflegern nicht durchgeführt werden. Zudem würde dadurch dem Seniorenheimbewohner sichtlich in die persönliche Freiheit eingeriffen, wenn in regelmäßigen Abständen Pfleger zur kurzen Kontrolle ins Zimmer kommen würden. Zum anderen ist die Gesundheit des Senioren von immenser Wichtigkeit. Viele Krankheiten, wie eine Erkältung oder eine Lungenetzündung, lassen sich vermeiden, wenn die Luftbeschaffenheit optimal für den Senior gesetzt ist.\\
Von dieser Motivation aus wird das im nächsten Kapitel beschriebene Ziel angestrebt.
\section{Ziel}
Das Ziel dieser Arbeit ist es, durch eine methodische Vorgehensweise einen ``perfekten'' Raum für Senioren zu schaffen. Der Raum wird dafür als ein regelbasiertes System betrachtet, das die optimale Umgebung für Senioren bilden wird.\\Den Angehörigen und Pflegern soll eine mobile Schnittstelle zur Verfügung gestellt werden, die ihnen dabei hilft, eine visuelle, aber auch eine sensorbasierte Kontrollinstanz für den Raum aufzubauen. Neben der Möglichkeit den Raum zu überprüfen, soll dem Nutzer gleichzeitig aus den erhalteten Sensordaten konkrete Verbesserungsvorschläge gemacht werden.\\
Somit können die Nutzer, auf dieser Benutzerschnittstelle aufbauend, eine effizientere Raumkontrolle und Verbesserung der Umgebung aufsetzen.
Außerdem sollen vom Benutzer Grenzwerte oder ein Werteverlauf selbstständig festgelegt werden können. Der Nutzer wird benachrichtigt, falls der entsprechende Wert sich nicht mehr im Toleranzbereich befindet. 
Folgende Aufgaben soll die App dabei abdecken:
\begin{itemize}
	\item Anzeige eines Live-Bild einer Kamera
	\item Anzeige der aktuellen Temperatur
	\item Anzeige der aktuellen Luftfeuchtigkeit
	\item Anzeige der Luftgaszusammensetzung
	\item Anzeige eines Status, ob ein Fester/Tür geöffnet oder geschlossen ist
	\item Anzeige eines zeitlichen Verlaufs der oben genannten Sensordaten
\end{itemize} 
Die verschiedenen Sensoren werden an einem Raspberry Pi angeschlossen und deren Werte zentral erfasst. Diese ermittelten Werte der Sensoren sollen anschließend über eine Schnittstelle von der App aufgerufen werden können. 
\\
Des Weiteren soll in der Arbeit ein Ausblick auf eine mögliche Erweiterung gegeben werden, wie sich Vorhersagen, auf Basis der bestehenden Daten, zu einzelnen Sensoren bestimmen lassen und der Benutzer dadurch profitieren kann. So kann beispielsweise im Fall der Luftqualität schon im Vorhinein auf Basis alter Daten eine Verschlechterung vorhergesagt werden, sodass Gegenmaßnahmen frühzeitig getroffen werden können.
\section{Struktur der Arbeit}
Zuerst ist eine Einarbeitung in die Thematik der regelbasierten Systeme, der Sensortechnik und der Nutzung eines Raspberry Pis notwendig. Darauf aufbauend werden die Anforderungen an das zu erstellende System erhoben. Um sinnvolle Anforderungen aufstellen zu können, muss definiert werden, was ein ``perfekter'' Raum für Senioren ist und dessen Vorteile. Anschließend wird ein übergreifender Anforderungskatalog erstellt. Mit Hilfe des Anforderungskataloges werden anschließend Technologien und Vorgehensweisen zur Erfüllung der Anforderungen verglichen und bewertet.\\
Bei der Auswahl der einzelnen benötigten Komponenten, wird als Grundlage festgesetzt eine mobile Applikation des Betriebssystemes Android zu erstellen. Auf dadurch entstehende Einschränkungen und Auswirkungen auf die Gesamtarchitektur wird durchgehend gesondert geachtet. 	