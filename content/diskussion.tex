\chapter{Fazit}
Mithilfe des implementierten System hat der Anwender die Möglichkeit seinen Raum zentral zu überwachen. Pfleger können durch die Applikation gezielter Maßnahmen ergreifen, um die Raumbedingungen zu verbessern. Dabei bekommen sie verschiedene Informationen zu den Eigenschaften der Raumbedingungen dargestellt. Basierend auf diesen Ergebnissen kann er Handlungen vornehmen, um die Bedingungen im Raum zu beeinflussen.\\
Durch die Regeldefintion kann der Pfleger individuelle Regeln für den Bewohner des Seniorenheims erstellen. Nur durch die indivudellen regeln lassen sich die optimalen Raumbedingungen für den Bewohner überwachen.\\
Die Überwachung über das System bietet noch Ungenauigkeiten, deshalb ersetzt es keinesfalls eine persönliche Kontrolle der Bewohner im Seniorenheim. Die Ungenauigkeiten kommen zum einen durch die Messabweichungen der Sensoren und zum anderen durch eine eventuelle falsche Datenübertragung zwischen Raspberry Pi und der mobilen Applikation. Das System bietet also lediglich eine Hilfe bei der Überwachung der Raumbedingungen. Den Pflegern muss das immer bewusst sein.\\
Eine mögliche Schwäche der Rule Engine ist der JavaScript Interpreter. Auch wenn dieser viele Vorteile mit sich bringt und aus dem Grund auch verwendet wurde, muss die Sicherheitslücke geschlossen werden, durch die Schadcode eingefügt werden kann. Das Eingabefeld des Dann-Feldes bei einer Regelerstellung wird nicht überprüft und lässt eine freie Texteingabe zu. 