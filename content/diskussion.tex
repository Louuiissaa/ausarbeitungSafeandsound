\chapter{Fazit}

Mithilfe des implementierten Systems hat der Anwender die Möglichkeit seinen Raum zentral zu überwachen. Pfleger können durch die Applikation gezielter Maßnahmen ergreifen, um die Raumbedingungen zu verbessern. Dabei bekommen sie verschiedene Informationen zu den Eigenschaften der Raumbedingungen übersichtlich dargestellt. Darüber hinaus werden Handlungsempfehlungen mit konkreten Aktionen gegeben, wie der Anwender die Raumbedingungen positiv verändern kann. 
 
Da das System an einigen Stellen noch Ungenauigkeiten aufweist, sollte es zunächst als eine zusätzliche Kontrollmöglichkeit der Raumbedingungen eingesetzt werden, jedoch eine persönliche Kontrolle der Bewohner im Seniorenheim nicht völlig ersetzen. Die Ungenauigkeiten kommen zum einen durch die Messabweichungen der Sensoren und zum anderen durch eine eventuelle falsche Datenübertragung zwischen Raspberry Pi und der mobilen Applikation. \\
Eine mögliche Schwäche der Applikation ist der JavaScript Interpreter der Rule Engine. Dieser bringt zwar viele Vorteile mit sich, kann aber erst ohne Bedenken eingesetzt werden, sobald die Sicherheitslücke, welche das Einfügen von Schadcode möglich macht, geschlossen wurde. Das Eingabefeld des Dann-Feldes bei einer Regelerstellung wird nicht überprüft und lässt eine freie Texteingabe zu, das zum Ausführen von Schadcode genutzt werden kann.\\
Ein weiterer Nachteil des implementierten Systems ist das Hinzufügen von weiteren Sensoren. Der Sensor muss individuell analysiert werden, um zu erkennen was die beste Anbindung ist. Der Nutzer müsste dafür, neben der physikalischen Anbindung an den Raspberry Pi, den Sensor in Node-RED hinzufügen. Je nach Sensor kann dort die Integration stark abweichen und eventuell einen hohen Mehraufwand mit sich bringen. Auch wenn die Rule Engine beim Hinzufügen neuer Sensoren nicht angepasst werden muss, muss die grafische Benutzeroberfläche um die Sensorenschnittstelle erweitert werden. Auch die Abbildung des Diagramms für den neuen Datentyp des hinzugefügten Sensors muss angepasst werden.\\
Um das implementierte System optimal in den Arbeitsalltag von Pflegekräften integrieren zu können, müssen definitiv noch Anpassungen getätigt werden, um den Aufwand bei Benutzung der Applikation so gering wie möglich zu gestalten. 
