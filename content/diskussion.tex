\chapter{Fazit}

Mithilfe des implementierten Systems hat der Anwender die Möglichkeit seinen Raum zentral zu überwachen. Pfleger können durch die Applikation gezielter Maßnahmen ergreifen, um die Raumbedingungen zu verbessern. Dabei bekommen sie verschiedene Informationen zu den Eigenschaften der Raumbedingungen übersichtlich dargestellt. Darüber hinaus werden Handlungsempfehlungen mit konkreten Aktionen gegeben, wie der Anwender die Raumbedingungen positiv verändern kann. 
 
Da das System an einigen Stellen noch Ungenauigkeiten aufweist, sollte es zunächst als eine zusätzliche Kontrollmöglichkeit der Raumbedingungen eingesetzt werden, jedoch eine persönliche Kontrolle der Bewohner im Seniorenheim nicht völlig ersetzen. Die Ungenauigkeiten kommen zum einen durch die Messabweichungen der Sensoren und zum anderen durch eine eventuelle falsche Datenübertragung zwischen Raspberry Pi und der mobilen Applikation. \\
Eine mögliche Schwäche der Rule Engine ist der JavaScript Interpreter. Dieser bringt zwar viele Vorteile mit sich, kann aber erst ohne Bedenken eingesetzt werden, sobald die Sicherheitslücke, welche das Einfügen von Schadcode möglich macht, geschlossen wurde. Das Eingabefeld des Dann-Feldes bei einer Regelerstellung wird nicht überprüft und lässt eine freie Texteingabe zu, das zum Ausführen von Schadcode genutzt werden kann.
