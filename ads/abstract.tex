%!TEX root = ../dokumentation.tex

\pagestyle{empty}

% Dieser deutsche Teil wird nur angezeigt, wenn die Sprache auf Deutsch eingestellt ist.
\renewcommand{\abstractname}{\langabstract} % Text für Überschrift

\begin{abstract}
Die Arbeit beschäftigt sich mit einem Sensorkontrollsystem im Umfeld von Seniorenheimen. Um ein Kameramodul und Sensoren anschließen zu können, wurde ein Raspberry Pi verwendet. Auf dessen Basis wurde zentral ein regelbasiertes System zur Verfügung gestellt. Es wurde analysiert wie Daten aus der Kamera und den Sensoren ausgelesen und verarbeitet werden können um anschließend Akteuren in geeigneter Weise zur Verfügung zu stellen. Den Akteuren werden Handlungsempfehlungen vorgeschlagen, damit sie die Raumbedingungen, welche vom Sensorkontrollsystem überwacht werden, optimal setzen können. \\
Das Ziel dieser Arbeit war es, ein System zur Verfügung zu stellen, dass dabei hilft, Aktionen zu tätigen, die optimal auf individuelle Bedürfnisse und die gegebenen Raumbedingungen abgestimmt sind.\\
Um das Ziel zu erreichen, wurde innerhalb dieser Arbeit evaluiert, was für Komponenten ein Sensorkontrollsystem mit grafischer Benutzeroberfläche innehaben sollte. Des Weiteren wurde betrachtet wie die einzelnen Komponenten im System aufgeteilt und umgesetzt werden sollten. Dabei wurde detailliert auf die Umsetzung der Komponenten eingegangen. Nach theoretischen Recherchearbeiten wurden verschiedene Lösungsmöglichkeiten dargestellt und auf deren Grundlage die mögliche Realisierung erläutert.
\end{abstract}


