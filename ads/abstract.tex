%!TEX root = ../dokumentation.tex

\pagestyle{empty}

% Dieser deutsche Teil wird nur angezeigt, wenn die Sprache auf Deutsch eingestellt ist.
\renewcommand{\abstractname}{\langabstract} % Text für Überschrift

\begin{abstract}
Die Arbeit beschäftigt sich mit einem Sensorkontrollsystem im Umfeld von Seniorenheimen. Um ein Kameramodul und Sensoren anzuschließen, wurde ein Raspberry Pi genutzt. Auf dessen Basis wurde zentral ein regelbasiertes System zur Verfügung gestellt. Es wurde analysiert, wie Daten aus der Kamera und den Sensoren ausgelesen und diese verarbeitet werden können, um anschließend Akteuren zur Verfügung zu stellen werden. Es wird sich innerhalb dieser Arbeit auf Menschen als Akteure beschränkt. Diese erhalten Handlungsempfehlungen, um die Raumbedingungen, das vom Sensorkontrollsystem überwacht wird, optimal zu setzen.\\
Das Ziel dieser Arbeit war es dementsprechend, ein System zur Verfügung zu stellen, dass dabei hilft, Aktionen zu tätigen, die optimal auf Bedürfnisse und die gegebenen Raumbedingungen abgestimmt sind.\\
Um das Ziel zu erreichen, wurde innerhalb dieser Arbeit evaluiert, was für Komponenten ein Sensorkontrollsystem mit grafischer Benutzeroberfläche braucht. Des Weiteren wurde betrachtet wie die einzelnen Komponenten im System aufgeteilt und umgesetzt sollten. Dabei wurde detailliert auf die Umsetzung der Komponenten eingegangen. Nach theoretischen Recherchearbeiten wurden verschieden Lösungsmöglichkeiten dargesetellt und auf deren Grundlage die optimale Realisierung ausgwählt.\\ 

Zum Schluss der Arbeit wird ein Ausblick gegeben, wie eine automatisierte Reaktion verschiedener Akteure auf die Sensordaten realisiert werden kann.
\end{abstract}


